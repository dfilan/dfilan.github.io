\documentclass[12pt]{article}
\usepackage{geometry}           
\geometry{letterpaper}
\usepackage{graphicx}
\usepackage{amssymb}
\usepackage{epstopdf}
\usepackage{amsmath}
\usepackage{amsthm}
\usepackage{mathtools}
\usepackage{pdfpages}
\usepackage{listings}
\usepackage{pgfplots}

\usepackage{tikz}
\usetikzlibrary{arrows,shapes}

\newcommand{\ts}[1]{\textsuperscript{#1}}
\newcommand{\tS}[1]{\textsubscript{#1}}
\newcommand{\R}{\mathbb{R}}
\newcommand{\N}{\mathbb{N}}
\newcommand{\U}{\mathcal{U}}

\title{Definitions in Topology}
\author{}
\date{}

\begin{document}
\bibliographystyle{alpha}
\maketitle

\vspace{-6ex}

A \textbf{topological space} is a set $X$ of \textbf{points} and a collection $\U \subseteq 2^X$ of \textbf{open sets} such that:
\begin{itemize}
\item $\emptyset, X \in \U$.
\item If $U, V \in \U$, then $U \cap V \in \U$.
\item For any set $I$ and any function $f: I \rightarrow \U$, $\bigcup_{i \in I} f(i) \in \U$.
\end{itemize}
Such an admissible $\U$ is called a \textbf{topology}. A set is \textbf{closed} if it is the complement of an open set, and \textbf{clopen} if it is both closed and open. A set $N$ is a \textbf{neighbourhood} of a point $x$ if there is some open set $U$ such that $U \subseteq N$ and $x \in U$. Alternatively, it is a neighbourhood of a set $E$ if there is some open set $U$ such that $E \subseteq U \subseteq N$. A point $x$ is a \textbf{limit point} of a set $E$ if every neighbourhood of $x$ contains a point in $E \setminus {x}$.

A sequence $(x_n)$ in $X$ \textbf{converges} to a point $x$ if for all open neighbourhoods $U$ of $x$ there exists $N \in \N$ such that for all $n \geq N$, $x_n \in U$.

If $X$ and $Y$ are topological spaces, the function $f: X \rightarrow Y$ is \textbf{continuous} if for any open set $U \subseteq Y$, its inverse image $f^{-1}(U) \subseteq X$ is also open. $f$ is a \textbf{homeomorphism} if it is invertible and if $f^{-1}$ is also continuous.

A topology $\U$ is \textbf{generated} by $B \subseteq 2^X$ if it is the smallest topology containing $B$. We say that the generating set $B$ is a \textbf{base} of the topology if also every point in $X$ is contained in an element of the base, and for every $B_1, B_2 \in B$, for every $x \in B_1 \cap B_2$ there exists some $B_3 \in B$ such that $x \in B_3 \subseteq B_1 \cap B_2$. A \textbf{local base} for a point $x$ is a collection of neighbourhoods of $x$ such that any other neighbourhood of $x$ contains an element of the base.

If $X$ is a topological space and $Y \subseteq X$, the \textbf{subspace topology} on $Y$ is $\{V \in Y \mid V = U \cap Y, U \subseteq X, \text{ $U$ open}\}$. If $X_i$ is a family of topological spaces for $i \in I$, then the \textbf{product space} $X = \prod_{i} X_i$ is the Cartesian product of the spaces $X_i$, with the topology generated by the sets $\prod_{i \in I} U_i$ where each $U_i$ is open in $X_i$ and $U_i \neq X_i$ for only finitely many $i$. If $\sim$ is an equivalence relation on $X$ then the \textbf{quotient space} $X / \sim$ is the set $\{[x] = \{y \in X \mid y \sim x\} \mid x \in X\}$ with the topology $\{U \subseteq X / \sim \, \, \mid \bigcup_{[x] \in U} \bigcup_{y \in [x]} y \text{ open in $X$}\}$.

A space $X$ is \textbf{Kolmogorov}, or \textbf{T\tS{0}}, if for every pair of distinct points at least one has a neighbourhood not containing the other. It is \textbf{Fr\'{e}chet}, or \textbf{T\tS{1}}, if each of the two points has a neighbourhood not containing the other. It is \textbf{Hausdorff}, or \textbf{T\tS{2}}, if the neighbourhoods can be made disjoint. It is \textbf{Urysohn}, or \textbf{T\tS{2.5}}, if the disjoint neighbourhoods can additionally be made closed. It is \textbf{completely Hausdorff} if there exists a continuous function $f: X \rightarrow [0,1]$ with $f(x) = 0$ and $f(y) = 1$. It is \textbf{regular} if for any closed set $C$ and any point $x \notin C$, $x$ and $C$ have disjoint neighbourhoods, and it is \textbf{regular Hausdorff} or \textbf{T\tS{3}} if it is regular and Hausdorff. It is \textbf{completely regular} if there is a continuous function $f: X \rightarrow \R$ such that $f(x) = 0$ and $f(Y) = \{1\}$, and \textbf{Tychonoff} or \textbf{T\tS{3.5}} if it is completely regular and Hausdorff. A space is \textbf{normal} if every pair of disjoint closed sets $C$ and $D$ have open neighbourhoods, it is \textbf{completely normal} if every subspace is normal, and it is \textbf{perfectly normal} if there exists a continuous function $f: X \rightarrow [0,1]$ such that $f^{-1}(\{0\}) = C$ and $f^{-1}(\{1\}) = D$. Spaces that are normal and Hausdorff are called \textbf{T\tS{4}}, spaces that are completely normal and Hausdorff are called \textbf{T\tS{5}}, and spaces that are perfectly normal and Hausdorff are called \textbf{T\tS{6}}.

A subset $E$ of $X$ is \textbf{sequentially open} if for all $x \in E$ and all sequences $(x_n)$ converging to $x$, there exists $N \in \N$ such that for all $n \geq N$, $x_n \in E$. A space is \textbf{sequential} if every sequentially open subset is open. A subset $E$ of $X$ is \textbf{dense} if every open set in $X$ has a non-empty intersection with $E$, and $X$ is \textbf{separable} if it has a countable dense subset. A space is \textbf{first-countable} if every point has a countable local base, it is \textbf{second-countable} if the topology has a countable base.

A \textbf{cover} of a space is a collection of sets whose union is the whole space, and a \textbf{sub-cover} is a sub-collection of a cover which itself is also a cover. A \textbf{refinement} of a cover is a new cover whose elements are all subsets of elements of the original cover. A cover is \textbf{locally finite} if every point in the space has a neighbourhood that intersects only finitely many sets in the cover.

A space is \textbf{Lindel\"{o}f} if every cover of open sets (or open cover) has a countable sub-cover, and is \textbf{compact} if the sub-cover is finite.  \textbf{Countably compact} only need have finite sub-covers for countable open covers. Spaces are \textbf{paracompact} if open covers have open refinements which are locally finite. A space is \textbf{$\sigma$-compact} if it has a countable cover by compact subspaces, and \textbf{locally compact} if every point has a compact neighbourhood. It is \textbf{compactly generated} if subset $A$ is closed iff for all compact subspaces $K$, $A \cap K$ is closed in $K$. A \textbf{sequentially compact} space is one where every sequence has a subsequence that converges. A \textbf{pseudocompact} space is one such that its image under any continuous function mapping it to $\R$ is bounded, and a \textbf{limit point compact} space is one where every infinite set has a limit point.

A space is \textbf{metrisable} if it is homeomorphic to a metric space, and \textbf{locally metrisable} if every point has a metrisable neighbourhood.

A \textbf{disconnected} space is one which has a non-trivial clopen subset, and a \textbf{connected} space is one which is not disconnected. A space $X$ is \textbf{totally disconnected} if it has no non-trivial connected subsets, and \textbf{totally separated} if for any points $x$ and $y$, there are disjoint open neighbourhoods $U$ of $x$ and $V$ of $y$ such that $X = U \cup V$. $X$ is \textbf{path-connected} if for any two points $x$ and $y$, there is a continuous function $f: [0,1] \rightarrow X$ such that $f(0) = x$ and $f(1) = y$, and \textbf{arc-connected} if $f$ is a homeomorphism between $[0,1]$ and $f([0,1])$. A space is \textbf{simply connected} if every pair of paths between two points can be continuously transformed into each other, more formally, if for any continuous map $f$ from the unit circle in $\R^2$ to $X$, there exists a continuous map $F$ from the unit disc in $\R^2$ to $X$ such that $F = f$ when restricted to the unit circle. A space is \textbf{hyperconnected} if no two non-empty open sets are disjoint. A \textbf{locally connected} space is one where every point has a local base of open connected sets, and a \textbf{locally path-connected} space is one where the local base is of open path-connected sets.

\end{document}

%%% Local Variables:
%%% mode: latex
%%% TeX-master: t
%%% End:
