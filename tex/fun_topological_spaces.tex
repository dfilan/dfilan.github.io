\documentclass[12pt]{article}
\usepackage{geometry}           
\geometry{letterpaper}
\usepackage{graphicx}
\usepackage{amssymb}
\usepackage{epstopdf}
\usepackage{amsmath}
\usepackage{amsthm}
\usepackage{mathtools}
\usepackage{pdfpages}
\usepackage{listings}
\usepackage{pgfplots}

\usepackage{tikz}
\usetikzlibrary{arrows,shapes}

\newcommand{\ts}[1]{\textsuperscript{#1}}
\newcommand{\tS}[1]{\textsubscript{#1}}
\newcommand{\R}{\mathbb{R}}
\newcommand{\N}{\mathbb{N}}
\newcommand{\U}{\mathcal{U}}

\title{Fun spaces to classify}
\author{}
\date{}

\begin{document}
\bibliographystyle{alpha}
\maketitle

\vspace{-6ex}

\begin{itemize}
\item Any set $X$ endowed with the \textbf{trivial topology}, where the only open sets are $\emptyset$ and $X$.
\item Any set $X$ endowed with the \textbf{discrete topology}, where all sets are open.
\item The \textbf{Sierpi\'{n}ski space} $\{0,1\}$ with topology $\{\emptyset, \{1\}, \{0,1\}\}$.
\item Any non-empty set $X$ containing a point $p$ endowed with the \textbf{particular point topology}, where the only open sets are the empty set and those sets that contain $p$.
\item $[0,1]$ with the standard topology.
\item $(0,1)$ with the standard topology.
\item $[0,1)$ with the standard topology (is this homeomorphic to $\R$?).
\item $[0,1]$ with the \textbf{either-or topology}, where a set is open if it either does not contain $\{0.5\}$ or it does contain $(0,1)$.
\item The circle.
\item $\N$ with the \textbf{cofinite topology}, where a set is open if it is empty or if its complement is finite.
\item $\R$ with the \textbf{cocountable topology}, where a set is open if it is empty or if its complement is countable.
\item The \textbf{topologist's sine curve}, $\{(x, \sin (1/x)) \mid x \in (0,1]\} \cup \{(0,0)\}$, with the subspace topology inherited from the Euclidean plane.
\item The \textbf{Sorgenfrey line}, which is $\R$ with the topology generated by the sets $[a,b)$.
\item The \textbf{Sorgenfrey plane}, which is the product of the Sorgenfrey line with itself.
\item The \textbf{long ray}, which is the cartesian product of the first uncountable ordinal $\omega_1$ with $[0,1)$. This set has the lexicographic order where $(a,b) < (c,d)$ if $a < c$ or if $a = c$ and $b < d$. It is endowed with the \textbf{order topology}, which is generated by the sets $\{x \mid a < x \}$ and $\{x \mid x < b\}$.
\item The \textbf{long line}, which is constructed from the long ray in the same way that $(-1,1)$ is constructed from $[0,1)$.
\item The \textbf{Tychonoff plank}, which is the product of the ordinal spaces $[0,\omega]$ and $[0, \omega_1]$.
\item The \textbf{infinite broom}, which is the subspace of the Euclidean plane consisting of all closed line segments joining the origin to $(1, 1/n)$ for all positive integers $n$, together with the interval $(0.5, 1]$ on the $x$-axis.
\item The \textbf{integer broom}, which is the subset of the Euclidean plane given by the polar coordinates $\{r = n \mid n \in \N_0\} \times \{\theta = 1/k \mid k \in \N, k \geq 1\} =: U \times V$. It has the product topology generated by the standard topology on $V$ and the topology on $U$ generated by the sets $\{n \mid a < n\}$.
\item The \textbf{Cantor set}.
\item The \textbf{Hilbert cube}, which is the topological product $\prod_{i=1}^n [0,1/n]$.
\end{itemize}

\end{document}

%%% Local Variables:
%%% mode: latex
%%% TeX-master: t
%%% End:
